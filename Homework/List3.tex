

\documentclass[12pt]{article}
\usepackage[paper=letterpaper,margin=2cm]{geometry}
\usepackage{amsmath}
\usepackage{amssymb}
\usepackage{amsfonts}
\usepackage{newtxtext, newtxmath}
\usepackage{enumitem}
\usepackage{titling}
\usepackage[colorlinks=true]{hyperref}

\setlength{\droptitle}{-6em}

% Enter the specific assignment number and topic of that assignment below, and replace "Your Name" with your actual name.
\title{Assignment 2, Solution to exercise 5/L3 Economathematics}
\author{Kacper Kinastowski}
\date{\today}

\begin{document}
\maketitle
\begin{enumerate}[leftmargin=\labelsep]

\item \textbf{Exercise description} 

\begin{enumerate}
    \item \textbf{Ex 5} Find the expression for $\Delta$ in the Black-Scholes market.

    \item \textbf{Ex 6} Find the expression for $\Gamma$ in the Black-Scholes market.

    \item \textbf{Ex 7} Find the expression for $\nu$ in the Black-Scholes market.

    \item \textbf{Ex 7} Find the expression for $\rho$ in the Black-Scholes market.
\end{enumerate}

\item \textbf{Greek letter definitions} In this section I present definitions of the greeks in finance \cite{Yu2013OnDO}

\begin{enumerate}
    \item \textbf{Delta} Let $\Delta$ be a rate of change of option value $V$ with respect to the change in the asset price $S_0$

    \begin{equation}
        \Delta = \frac{\partial V}{\partial S_0}
    \end{equation}
    
    In the next steps we are going to prove that for European Call Option
    
    \begin{equation}
        \Delta_C = \frac{\partial C}{\partial S_0} = \Phi(d_1)
    \end{equation}

    \item \textbf{Gamma} Let $\Delta$ be a rate of change of $\Delta$ with respect to the change in the asset price $S_0$. In other words

    \begin{equation}
        \Gamma = \frac{\partial \Delta}{\partial S_0} 
        = \frac{\partial^2 V}{\partial S_0^2}
    \end{equation}

    In the next steps we are going to prove that for European Call Option
    
    \begin{equation}
        \Gamma_C = \frac{\partial \Delta_C}{\partial S_0}  =
        \frac{\partial^2 C}{\partial S_0^2} =
        \frac{1}{S_0}\frac{\varphi(d_1)}{\sigma  \sqrt{T}}
    \end{equation}

    \item \textbf{Vega} Let $\nu$ be a rate of change of option value $V$ with respect to the change in the volatility $\sigma$

    \begin{equation}
        \nu = \frac{\partial V}{\partial \sigma}
    \end{equation}

    with specific case for BS European Call Option:

    \begin{equation}\label{vega_c}
        \nu_C = \frac{\partial C}{\partial \sigma} = \varphi(d_1) S_0 \sqrt{T} 
    \end{equation}

    \item \textbf{Rho} Let $\rho$ be a rate of change of option value $V$ with respect to the change in the interest rate $r$

    \begin{equation}
        \rho = \frac{\partial V}{\partial r}
    \end{equation}

    with specific case for BS European Call Option:

    \begin{equation}\label{rho_c}
        \rho_C = \frac{\partial C}{\partial r} 
        = T K e^{-rT} \Phi(d_2)
    \end{equation}

    
    
\end{enumerate}




\item \textbf{Black-Scholes Model} According to Black-Scholes (BS) formula price of the European Call Option $C$ and price of the European Put Option $P$ at time $t = 0$ are given by the equations \cite{davis2010black}


\begin{align}\label{pc}
     C = \Phi(d_1)S_0 - \Phi(d_2)Ke^{-rT},\\
     P = -\Phi(-d_1)S_0 - \Phi(d_2)Ke^{-rT},
\end{align}

with coefficients:
\begin{equation}
\begin{split}\label{d}
     d_1 &= \frac{1}{\sigma\sqrt{T}}\left[\log\left(\frac{S_t}{K}\right) + \left(r + \frac{\sigma^2}{2}\right)(T)\right], \\
     d_2 &= d_1 - \sigma\sqrt{T}. \\
\end{split}
\end{equation}

In \eqref{pc} $\Phi(d)$ stands for standard, normal cumulative distribution function (CDF), so it's probability density function (PDF) can be written as

\begin{equation}\label{pdf}
\frac{\partial \Phi(d)}{\partial d} = \varphi(d) = 
\frac{1}{\sqrt{2\pi}}  e^{-d^2/2}.
\end{equation}

Definitions of $C$ and $P$ are related to each other via so called put-call parity

\begin{equation}
    P = C + K e^{-rT}
\end{equation}

\item \textbf{Important identities}

In this section i provide proofs for some important identities that are used in solution.

\begin{enumerate}

    \item Relation between derivatives of $d_1$ and $d_2$ with respect to $S_0$
    \begin{equation}\label{help1}
        \frac{\partial d_1}{\partial S_0} = \frac{\partial d_2}{\partial S_0}.
    \end{equation}
    \textbf{Proof}: Just by looking at \eqref{d} it is clear that \eqref{help1} is true.  $\blacksquare$

     \item Relation between derivatives of $d_1$ and $d_2$ with respect to volatility term $\sigma$
    \begin{equation}\label{help2}
        \frac{\partial d_1}{\partial \sigma} = \frac{\partial d_2}{\partial \sigma} + \sqrt{T}.
    \end{equation}
    \textbf{Proof}: Differentiating both sides of \eqref{d} with respect to $\sigma$ leads to \eqref{help2}.  $\blacksquare$

    \item Relation between PDFs $\varphi(d_1)$ and $\varphi(d_2)$
    \begin{equation}\label{identity1}
        \varphi(d_2) =\frac{S_0}{K} e^{rT} \varphi(d_1)
    \end{equation}
    \textbf{Proof}: It is trival that difference of squares
    
    \begin{equation}
        d_2^2 - d_1^2 = (d_2-d_1)(d_2 + d_1).
    \end{equation}
    
    Using the relations given by \eqref{d} one can obtain:

    \begin{equation}
    \begin{split}\label{identity0}
        d_1^2 - d_2^2 &=  - \sigma \sqrt{T} \left( 2 d_1 - \sigma \sqrt{T} \right) \\
        &=  - \sigma \sqrt{T} \left( \frac{2}{\sigma\sqrt{T}}\left[\log\left(\frac{S_0}{K}\right) + \left(r + \frac{\sigma^2}{2}\right)T\right]  - \sigma \sqrt{T} \right)\\
        &=  -2 \left(\log\left(\frac{S_0}{K}\right) + \left(r + \frac{\sigma^2}{2}\right)T  - \sigma^2 T \right) \\
        &=  -2 \left( \log\left(\frac{S_0}{K}\right) + 2rT + \sigma^2 T  - \sigma^2 T \right)\\
        &=  -2 \left( \log\left(\frac{S_0}{K}\right) + rT \right).\\
    \end{split}
    \end{equation}

    By using the definiton of PDFs \eqref{pdf} and calculating $\log(\varphi(d_1)/ \varphi(d_2))$:

    \begin{equation}
    \begin{split}\label{identity2}
        \log \left( \frac{\varphi(d_1)}{\varphi(d_2)} \right)
        & = \log \left( e^{-d_1^2/2} \right) - \log \left( e^{-d_2^2/2} \right) \\
        & = \frac{1}{2} \left( d_2^2 - d_1^2 \right).
    \end{split}
    \end{equation}

    Combining equations \eqref{identity0} and \eqref{identity2}

    \begin{equation}
    \begin{split}
    \log \left( \frac{\varphi(d_1)}{\varphi(d_2)} \right) = -\log\left(\frac{S_0}{K}\right) - rT, \\
    \frac{\varphi(d_1)}{\varphi(d_2)} = \frac{K}{S_0} e^{-rT}, \\
    \varphi(d_2) =\frac{S_0}{K} e^{rT} \varphi(d_1) ,
    \end{split}
    \end{equation}

    what ends the proof. $\blacksquare$
    
\end{enumerate}






\item \textbf{Solutions}

\begin{itemize}
    \item \textbf{Ex 5} Derivation of $\Delta$

We obtain $\Delta_C$ by calculating derivative of \eqref{pc} with respect to $S_0$. Index 'C' informs that this parameter is calculated for European Call Option. By using the fact \eqref{help1} we obtain


\begin{equation}
\begin{split}
 \label{delta_c}
        \Delta_C &= \frac{\partial}{\partial S_0} 
    \left[ \Phi(d_1)S_0 - \Phi(d_2)Ke^{-rT} \right] \\
        &= \Phi(d_1) + 
    \frac{\partial \Phi(d_1) }{\partial d_1} \frac{\partial d_1}{\partial S_0} S_0
    - \frac{\partial \Phi(d_2) }{\partial d_2} \frac{\partial d_2}{\partial S_0}
    Ke^{-rT} \\
        &= \Phi(d_1) + 
     \frac{\partial d_1}{\partial S_0} \left[ 
     \frac{\partial \Phi(d_1) }{\partial d_1}S_0
     - \frac{\partial \Phi(d_2) }{\partial d_2}Ke^{-rT}
     \right] \\
      &= \Phi(d_1) + 
     \frac{\partial d_1}{\partial S_0} \left[ 
     \varphi(d_1)S_0
     -\varphi(d_2)Ke^{-rT}
     \right]. \\   
\end{split}
\end{equation}

Now substituting \eqref{identity1} to \eqref{delta_c} we obtain the explicit form of $\Delta_C$

\begin{equation}
\begin{split}
 \label{delta_c1}
        \Delta_C &=  \Phi(d_1) + 
     \frac{\partial d_1}{\partial S_0} \left[ 
     \varphi(d_1)S_0
     -\varphi(d_2)Ke^{-rT}
     \right]\\   
     &=  \Phi(d_1) + 
     \frac{\partial d_1}{\partial S_0} \left[ 
     \varphi(d_1)S_0
     -\frac{S_0}{K} e^{rT} \varphi(d_1) Ke^{-rT}
     \right]\\  
     &=  \Phi(d_1) + 
     \frac{\partial d_1}{\partial S_0} \left[ 
     \varphi(d_1)S_0
     -\varphi(d_1) S_0
     \right]\\ 
     &=  \Phi(d_1) \blacksquare
\end{split}
\end{equation}

\item \textbf{Ex 6} Derivation of $\Gamma$

We obtain $\Gamma_C$ by calculating derivative of \eqref{delta_c1} with respect to $S_0$. Index 'C' informs that this parameter is calculated for ECO

\begin{equation}
\begin{split}\label{gamma}
        \Gamma_C &= \frac{\partial \Delta_C}{\partial S_0}
        = \frac{\partial \Phi(d_1)}{\partial S_0} \\
        &= \frac{\partial  \Phi(d_1)}{\partial d_1} 
        \frac{\partial  d_1}{\partial S_0} 
        = \varphi(d_1) \frac{1}{\sigma \sqrt{T}}\frac{\partial }{\partial S_0}  \left[ \log\left(\frac{S_0}{K}\right) \right] \\
        &= \frac{1}{S_0}\frac{\varphi(d_1)}{\sigma  \sqrt{T}} \blacksquare\end{split}
\end{equation}

\item \textbf{Ex 7} Derivation of $\nu$

By the definition \eqref{vega_c} we have:

\begin{equation}
\begin{split}\label{vega_c1}
        \nu_C &= \frac{\partial C}{\partial \sigma}
        = \frac{\partial }{\partial \sigma} 
        \left[ \Phi(d_1)S_0 - \Phi(d_2)Ke^{-rT} \right] \\
        &= \frac{\partial \Phi(d_1) }{\partial d_1} \frac{\partial d_1}{\partial \sigma} S_0
    - \frac{\partial \Phi(d_2) }{\partial d_2} \frac{\partial d_2}{\partial \sigma}
    Ke^{-rT}\\
        &= \varphi(d_1) \frac{\partial d_1}{\partial \sigma} S_0
    - \varphi(d_2) \frac{\partial d_2}{\partial \sigma}
    Ke^{-rT}\\
\end{split}
\end{equation}

Now using identities \eqref{help2} and \eqref{identity1}

\begin{equation}
\begin{split}\label{vega_c1}
    \nu_C &= \varphi(d_1) S_0 \left( \frac{\partial d_2}{\partial \sigma} + \sqrt{T} \right)
    - \frac{S_0}{K} e^{rT} \varphi(d_1) \frac{\partial d_2}{\partial \sigma}
    Ke^{-rT}\\
    &= \varphi(d_1) S_0 \sqrt{T} 
    + \varphi(d_1) S_0 \frac{\partial d_2}{\partial \sigma}
    - \varphi(d_1) S_0 \frac{\partial d_2}{\partial \sigma} \\
    &= \varphi(d_1) S_0 \sqrt{T}. \blacksquare
\end{split}
\end{equation}

\end{itemize}
\end{enumerate}

\bibliography{bib} 
\bibliographystyle{ieeetr}


\end{document}
